% Select one class
%\documentclass{pucthesis}		% For DVI
% \documentclass[pdftex]{pucthesis}	% For pdfLaTeX
% \documentclass[spanish]{pucthesis}		% For DVI, in spanish
\documentclass[pdftex,spanish]{pucthesis}	% For pdfLaTeX, in spanish

%%%%%%%%% Packages %%%%%%%%%

% Floats
\usepackage{graphicx}
\usepackage{float}
\floatstyle{boxed}
\restylefloat{figure}
\usepackage{subfig}

% Math packages
\usepackage{amsmath}
\usepackage{amsfonts}
\usepackage{amssymb}

% Closest font to Times New Roman
\usepackage{times}

% To make pretty tables
\usepackage{booktabs}
\usepackage{multirow}
\usepackage{threeparttable}
\usepackage{caption}


% To avoid underfull errors in the bibliography
\usepackage{etoolbox}
\apptocmd{\sloppy}{\hbadness 10000\relax}{}{}

% To make cites and references
\usepackage[hidelinks,pdfusetitle,pdfdisplaydoctitle]{hyperref}
\usepackage{csquotes}
\usepackage[style=apa,sortcites=true,sorting=nyt,backend=biber]{biblatex}
% \usepackage{doi}
% \renewcommand{\doitext}{}

\usepackage{glossaries}
\makeglossaries
\newglossaryentry{order}{
	name={order},
	description={A request made by a customer through the Pedidos Ya platform for delivery}
}

\newglossaryentry{failrate}{
	name={fail rate},
	description={The percentage of orders that could not be completed due to various reasons}
}

\newglossaryentry{accrate}{
	name={acceptance rate},
	description={The percentage of delivery requests accepted by couriers}
}

\newglossaryentry{blackstores}{
	name={black stores},
	description={Warehouses or fulfillment centers used for fast order processing, typically not open to the public}
}


%--------- NEW ENVIRONMENTS --------- You are free to remove or use it
\newtheorem{definition}{\bf Definition}[chapter]
\newtheorem{property}{Property}[chapter]
\newtheorem{claim}{Claim}[chapter]
\newtheorem{lemma}{\bf Lemma}[chapter]
\newtheorem{proposition}{Proposition}[chapter]
\newtheorem{theorem}{\noindent \bf Theorem}[chapter]
\newtheorem{corollary}{\bf Corollary}[chapter]
\newtheorem{pf}{Proof}[chapter]
\newtheorem{example}{\bf Example}[chapter]
\newtheorem{remark}{Remark}[chapter]

\makeatletter
  \setlength{\beftitle}{105\p@\@plus24\p@}
  \setlength{\afttitle}{65\p@}
\makeatother

\addbibresource{Thesis.bib}

% ----- THEME ----  (Optional)
% Tokyo Night Theme Colors
\usepackage{xcolor}

% Define Tokyo Night color palette
\definecolor{tokyoNightBg}{HTML}{1a1b26}
\definecolor{tokyoNightFg}{HTML}{a9b1d6}
\definecolor{tokyoNightBorder}{HTML}{3d59a1}
\definecolor{tokyoNightBlue}{HTML}{7aa2f7}
\definecolor{tokyoNightCyan}{HTML}{7dcfff}
\definecolor{tokyoNightPurple}{HTML}{bb9af7}
\definecolor{tokyoNightGreen}{HTML}{9ece6a}
\definecolor{tokyoNightRed}{HTML}{f7768e}
\definecolor{tokyoNightOrange}{HTML}{ff9e64}
\definecolor{tokyoNightYellow}{HTML}{e0af68}
\definecolor{tokyoNightComment}{HTML}{565f89}

% Apply theme to document elements
\pagecolor{tokyoNightBg}
\color{tokyoNightFg}

% Hyperlinks styling
\usepackage{hyperref}
\hypersetup{
    colorlinks=true,
    linkcolor=tokyoNightBlue,
    filecolor=tokyoNightGreen,
    urlcolor=tokyoNightCyan,
    citecolor=tokyoNightPurple
}

% Section headings styling
% \usepackage{sectsty}
% \sectionfont{\color{tokyoNightBlue}}
% \subsectionfont{\color{tokyoNightPurple}}
% \subsubsectionfont{\color{tokyoNightCyan}}


% Optional: Code listings with Tokyo Night theme
\usepackage{listings}
\lstset{
    backgroundcolor=\color{tokyoNightBg},
    basicstyle=\color{tokyoNightFg}\ttfamily\small,
    keywordstyle=\color{tokyoNightBlue},
    commentstyle=\color{tokyoNightComment},
    stringstyle=\color{tokyoNightGreen},
    numberstyle=\tiny\color{tokyoNightComment},
    identifierstyle=\color{tokyoNightFg},
    frame=single,
    rulecolor=\color{tokyoNightBorder},
    showstringspaces=false
}
% Table styling
\usepackage{colortbl}

% Bibliography styling
% \usepackage[backend=biber, style=numeric]{biblatex}
% \renewcommand*{\bibfont}{\color{tokyoNightFg}}
% ---- Fin del preambulo ----

\begin{document}
\arrayrulecolor{tokyoNightBorder}

\mdate{Month Day, Year}         % date manuscript changed
\version{1}                                     % manuscript version #

\title[Informe de practica: Pedidos Ya]
{\bf Informe de practica: Pedidos Ya}
\author[Jorge Troncoso Morales]{Jorge Troncoso Morales}

\address                    {}
\email                      {jorge.troncoso@alu.ucm.cl}

\facultyto                  {the School of Engineering}
%\department                {}
\faculty                    {FACULTAD DE INGENIER\'IA}
\degree                     {Master of Science in Engineering}
\advisor                    {Ivan Merino Rodrigez}
\committeememberA           {Member A}
\guestmemberA               {Member B}
\ogrsmember                 {Member C}
\subject                    {Structural Engineering}
\date                       {Marzo 2025}
\copyrightname              {Jorge Troncoso Morales}
\copyrightyear              {2025}
% \dedication               {Gratefully to my parents and siblings}


\NoChapterPageNumber
\pagenumbering{roman}
\maketitle


%\newpage
%\thispagestyle{empty}

%%%%%%%%%%%%%%%%%%
%          page v & up ---                      %
%            Table of contents              %
%            List of figures                     %
%            List of tables                      %
%%%%%%%%%%%%%%%%%%

\pdfbookmark{\contentsname}{toc}
\tableofcontents
\phantomsection \label{listoffigures}
\listoffigures
\phantomsection \label{listoftables}
\listoftables
\cleardoublepage

%%%%%%%%%%%%% TEXT  OF THESIS %%%%%%%%%%%%%

\pagenumbering{arabic}
\chapter[DESCRIPCION GENERAL DE LA EMPRESA]{descripcion general de la empresa}
El presente informe tiene como propósito documentar la experiencia adquirida durante la práctica profesional realizada en PedidosYa Envios desde el 1 de Agosto 2024 hasta el 31 de Octubre 2025 en Las Condes, Santiago, en el cargo de Courier Business Intern. Se detalla el contexto organizacional, las funciones desempeñadas, el impacto de las actividades en el desarrollo de competencias profesionales y las oportunidades de mejora identificadas.

La práctica profesional se llevó a cabo en PedidosYa Envios, una empresa del sector courier especializada en logística y distribución de última milla. Durante un período de tres meses, el rol desempeñado se enfocó en tareas de análisis de datos, operaciones y business intelligence (BI) con el objetivo de optimizar procesos y mejorar la eficiencia operativa de la empresa.

Esta experiencia represento un gran impacto en la mejora de las competencias profesionales, permitiendo la aplicación de herramientas de análisis de datos, la familiarización con metodologías de optimización en el sector logístico y el desarrollo de habilidades de comunicación y toma de decisiones. Además, proporcionó una visión práctica del impacto de la ingeniería industrial en un entorno dinámico como el de la logística de última milla.

En los siguientes capítulos, se presentará un análisis detallado de la organización, las funciones realizadas, su relación con el perfil de Ingeniero Civil Industrial, así como propuestas de mejora para optimizar procesos dentro de la empresa.


\chapter[ROLES Y RESPONSABILIDADES DE LA PRACTICA]{Roles y responsabilidades de la practica} \label{ch1}
\section{Descripción General de la Posición: Courier Business Intern \label{sec:sec1}}
El rol de \textit{Courier Business Intern} consiste en apoyar las labores del área de Courier Business, participando activamente en tareas de análisis de datos, operaciones y business intelligence. Este puesto complementa el trabajo del Sr. Analyst, colaborando en la recolección, análisis y presentación de información, así como en la implementación de soluciones operativas específicas para PedidosYa Envíos.

\section{Responsabilidades Clave \label{sec:sec2}}
Aunque el enfoque principal del rol es el análisis de datos y la toma de decisiones basadas en información, las responsabilidades se llevan a cabo en coordinación con el equipo de Courier Business. Es importante señalar que PedidosYa cuenta con equipos especializados (BI, Operaciones y Logística) para abordar estos desafíos de forma integral; sin embargo, para tareas y problemas específicos, se requiere una solución más focalizada dentro del equipo de Courier Business.

\subsection{Análisis de Datos y Reporte}
\begin{itemize}
  \item \textbf{Recolección y Procesamiento de Datos:} Extraer y organizar datos relevantes para evaluar el desempeño de las operaciones.
  \item \textbf{Búsqueda de Insights:} Identificar tendencias y oportunidades de mejora a partir de los datos analizados.
  \item \textbf{Generación y Presentación de Reportes:} Elaborar informes detallados que respalden la toma de decisiones estratégicas, en colaboración con el Sr. Analyst.
\end{itemize}

\subsection{Soporte en Operaciones}
\begin{itemize}
  \item \textbf{Tareas Operativas Específicas:} Realizar actividades no críticas como renombrar partners, activar cuentas o gestionar partners no clave (Key Accounts).
  \item \textbf{Implementación de Soluciones:} Colaborar en la aplicación de mejoras operativas que requieran un enfoque personalizado y granular.
\end{itemize}

\subsection{Presentación y Toma de Decisiones}
\begin{itemize}
  \item \textbf{Reuniones Semanales:} Preparar y exponer semanalmente reportes e insights ante el equipo y el Manager.
  \item \textbf{Sugerencias Estratégicas:} Proponer recomendaciones basadas en los análisis realizados para optimizar las operaciones y procesos.
\end{itemize}

\section{Herramientas y Tecnologías Utilizadas \label{sec:sec3}}
Para llevar a cabo las tareas descritas, PedidosYa utiliza diversas soluciones tecnológicas que facilitan la gestión de datos y la comunicación interna. Entre las principales herramientas se encuentran:

\begin{enumerate}
  \item \textbf{Jira:} Sistema de gestión de tickets utilizado para el escalado de incidencias de clientes y agentes.\footnote{Más información en: \url{https://www.atlassian.com/software/jira}}
  \item \textbf{Slack:} Plataforma de comunicación interna que permite una interacción fluida entre miembros de diferentes equipos.\footnote{Más información en: \url{https://slack.com/}}
  \item \textbf{Google Cloud Platform (GCP):} Infraestructura para la gestión de recursos tecnológicos y de datos.\footnote{Más información en: \url{https://cloud.google.com/gcp}}
  \item \textbf{BigQuery:} Servicio de almacenamiento y gestión de bases de datos SQL, parte de GCP, que facilita la extracción y análisis de información.\footnote{Más información en: \url{https://cloud.google.com/bigquery}}
  \item \textbf{Locker Studio:} Herramienta integrada en GCP para la creación de dashboards y consultas simples.
  \item \textbf{JupyterHub:} Plataforma IDE en línea para escribir código, principalmente en Python, utilizando librerías como Pandas y Matplotlib para la manipulación y visualización de datos.\footnote{Más información en: \url{https://jupyter.org/hub}}
  \item \textbf{Google Spreadsheets:} Aplicación similar a Excel para la creación de hojas de cálculo y tablas dinámicas, con funcionalidades de colaboración en tiempo real.\footnote{Más información en: \url{https://www.google.com/sheets/about/}}
\end{enumerate}




\chapter[PROYECTOS Y TAREAS]{Proyectos y tareas} \label{ch2}
\section{Tareas}
\subsection{Revision de casos Jira}
Pedidos Ya cuenta con un sistema de tickets dirigido tanto a clientes (B2B) como a consumidores (C2C). El servicio de PedidosYa Envios opera en ambos segmentos, por lo que se solicita frecuentemente el contacto con agentes de servicio para resolver incidencias en las órdenes. Para atender estos casos se sigue un Procedimiento Operativo Estándar (SOP) que define las acciones a seguir; sin embargo, cuando se presentan situaciones ambiguas, el caso se escala directamente al equipo de Courier para su revisión.

Esta tarea, una de las más rutinarias realizadas, consistía en revisar el historial de \textit{chat} acompañado de imágenes para comprender la situación. Se trataba de una gestión operativa enfocada en determinar si las órdenes fallidas (Fail Rate) debían ser compensadas o denegadas, requiriendo mayor criterio y objetividad que desafíos técnicos.

Es importante señalar que, aunque esta tarea es rutinaria y no implica una responsabilidad crítica, una gestión inadecuada podría generar fricción con consumidores o clientes B2B. Además, esta actividad se llevó a cabo durante las primeras tres semanas del internado, para posteriormente ser transferida a un equipo especializado en la sede central de Uruguay.

\subsection{Analisis de datos}

El area Courier de PedidosYa suele trabajar con KPI's similares a los de PedidosYa en general, como pueden ser:

\begin{itemize}
	\item Fail Rate \eqref{eq:FR}: Para medir la tasa de ordenes que no suelen ser exitosas. \textbf{Se busca reducir.}

	      \begin{equation} \label{eq:FR}
		      \text{Fail Rate} = \frac{\text{Rejected Orders}}{\text{Total Orders}} \times 100\%
	      \end{equation}

	\item Acceptance Rate \eqref{eq:AR}: Cuando una orden es emitida, se envia una notifiacion a los repartidores o \textit{Riders}, dicha orden puede ser aceptada o rechazada. El Acceptance Rate mide la tasa con la que los repartidores aceptan las ordenes. \textbf{Se busca aumentar.}
	      \begin{equation} \label{eq:AR}
		      \text{Acceptance Rate} = \frac{\text{Riders Accept}}{\text{Riders Notified}} \times 100\%
	      \end{equation}
	\item Vendor Late \eqref{eq:VL}: Una orden tiene varias fases desde que se emite la orden hasta que es entregada por el rider, uno de los mas importantes es el Vendor Late, el cual mide el tiempo que pasa desde el que el Rider llega al local y se le es entregada la orden. \textbf{Se busca reducir.}
	      \begin{equation}\label{eq:VL}
		      \text{Vendor Late} = \text{Pick up datetime} - \text{Rider Arrival datetime}
	      \end{equation}
\end{itemize}

Teniendo estos KPI como referencia, fue necesario analizar los datos disponibles en PedidosYa. Estos se encuentran alojados en BigQuery, por lo que su acceso requirió permisos específicos y la ejecución de consultas (\textit{queries}) en SQL.

El Fail Rate puede estar influenciado por múltiples factores. Por razones de confidencialidad, las variables categóricas relacionadas con estas problemáticas serán presentadas con nombres genéricos. El dataset utilizado en las tareas rutinarias de análisis de datos contenía los siguientes campos:

\begin{table}[H]
	\caption{Dataset de órdenes registradas}
	\raggedright
	\label{tab:main_dataset}
	\begin{threeparttable}
		\begin{tabular}{lc}
			\toprule
			\textbf{Field}         & \textbf{Datatype} \\
			\midrule
			order\_id              & Integer           \\
			partner\_id            & Integer           \\
			date                   & Datetime          \\
			status\_BO             & String            \\
			partner                & String            \\
			partner\_or\_franchise & String            \\
			reason\_text           & String            \\
			globalCode             & Integer           \\
			declared\_value        & Float             \\
			\bottomrule
		\end{tabular}
		\begin{tablenotes}[para]
			\small
			\textit{Nota:} Los tipos de datos estan nombrados con sus nombres genericos.
		\end{tablenotes}
	\end{threeparttable}
\end{table}

Como Courier Business Intern, mi principal labor fue analizar los datos para mantener el Fail Rate en valores tolerables para cada partner e identificar cuales son las principales causas, determinado por \mintinline[style=default, fontsize=\small]{python}{globalCode}.

Este dataset (Tabla \ref{tab:main_dataset}) funciona como una pseudo tabla de hechos, ya que contiene el registro de todas las órdenes realizadas. La consulta SQL utilizada para obtener estos datos generalmente se limitaba a extraer las órdenes del mes en curso.

A partir de esta información, se realizaba un análisis diario para identificar las incidencias con mayor impacto en el Fail Rate, determinar qué \textit{partners} y locales eran los más afectados, y tomar medidas en consecuencia. Si se trataba de una problemática de rápida resolución, se implementaban acciones inmediatas. En otros casos, cuando un \textit{partner} representaba un alto costo logístico o afectaba significativamente las operaciones del área Courier, esta información se trasladaba a los Account Managers del equipo, quienes podían utilizarla como punto de presión en futuras negociaciones.


% Caso 1
Este último aspecto resultó clave en el caso de un \textit{partner} específico—al que llamaremos \textbf{Partner A}—que presentaba un número inusualmente alto de incidencias del \textbf{tipo X}, lo que incrementaba su Fail Rate. Tras un análisis detallado, se identificó que la causa principal era la negativa del \textit{partner} a utilizar el sistema de \textbf{PIN} en sus órdenes, lo que no solo afectaba el Fail Rate, sino que también impactaba la experiencia de los repartidores. Como resultado, se informó a los Account Managers para negociar la adopción del PIN o, en su defecto, un aumento en las tarifas.

Por otro lado, si bien algunas incidencias tenían una interpretación y causa evidentes, en muchos casos presentaban cierto grado de ambigüedad. Para identificar la causa real del aumento en el Fail Rate, se aplicaron técnicas de \textbf{regresión}, con el objetivo de determinar si existía una relación \textit{lineal} entre las tasas de incidencia a lo largo del tiempo. Para ello, se utilizaron funciones de agregación sobre la columna categórica \mintinline[style=default,fontsize=\small]{python}{globalCode} (Tabla \ref{tab:main_dataset}).

Este enfoque permitió analizar la variación de todos los tipos de incidencias en un período determinado y, a partir de ello, establecer una posible causalidad entre el aumento del Fail Rate y problemas reales inferidos mediante esta técnica de \textit{feature engineering}.

En definitiva, el análisis de datos se enfocaba principalmente en monitorear los KPI, pero especialmente el Fail Rate en distintos períodos de tiempo (días, semanas o meses). Para ello, se aplicaban funciones de agregación, feature engineering y técnicas de correlación estadística, con el fin de interpretar e inferir las causas de las variaciones en este indicador.


\subsection{Presentación semanal de insights}

Una de las tareas más desafiantes fue la presentación semanal de insights producto del análisis de datos. Esta tarea se realizaba todos los lunes de manera presencial en una reunión denominada \textit{Weekly}, en la que participaban todos los miembros del equipo: el \textbf{Manager}, dos \textbf{Account Managers}, dos \textbf{Key Account Manager}, el \textbf{Sr. Analyst} y, ocasionalmente, \textbf{managers} o analistas de otras áreas. Durante la reunión, cada integrante exponía y comunicaba al equipo información relevante sobre operaciones, aspectos financieros, estado de los partners e insights encontrados en los análisis.

En mi rol como \textbf{intern}, fui responsable de exponer hallazgos relacionados con el análisis del Fail Rate. Esto requería sintetizar datos que, aunque para mí eran claros, podían resultar confusos en el contexto de la reunión. Por ello, mi presentación se centraba en torno a tres puntos clave:

\begin{enumerate}
	\item \textbf{Variación de la tasa de Fail Rate entre los principales partners en un Line Plot WoW (Week over Week):} Se identificaban los partners con mayor aumento o disminución en su Fail Rate, explicando las posibles causas detrás de estas fluctuaciones.
	\item \textbf{Variación de la tasa de tipos de incidencias en un HeatMap WoW:} Se analizaba la evolución semanal de los diferentes tipos de incidencias, buscando patrones o anomalías en los datos.
	\item \textbf{Flujo de órdenes de los principales partners y su destino en función de las incidencias en un diagrama Sankey:} Se visualizaba cómo se distribuían las órdenes entre los distintos tipos de incidencias, proporcionando un panorama detallado de los cuellos de botella o puntos críticos en la operación.
\end{enumerate}

Además de la presentación de datos, otro aspecto clave de esta tarea fue la interpretación y comunicación efectiva de los resultados. Para garantizar que los insights fueran comprensibles y accionables, empleé las siguientes estrategias:

\begin{itemize}
	\item \textbf{Uso de visualizaciones efectivas:} Se priorizó el uso de gráficos intuitivos y fáciles de interpretar, evitando sobrecargar las presentaciones con información excesiva.
	\item \textbf{Enfoque en la relevancia del negocio:} En cada insight, se destacaba su impacto en la operación y su relevancia para la toma de decisiones estratégicas.
	\item \textbf{Propuestas de acción basadas en datos:} Se sugerían medidas correctivas o preventivas en función de los hallazgos, permitiendo que el equipo tomara decisiones informadas.
	\item \textbf{Feedback y aprendizaje continuo:} Tras cada presentación, recibía comentarios del equipo, lo que me permitió mejorar la claridad y enfoque de mis análisis en reuniones posteriores.
\end{itemize}

El desarrollo de esta actividad fortaleció mis habilidades en la comunicación de datos, permitiéndome traducir información técnica en insights accionables para la empresa. Asimismo, me permitió trabajar en un entorno colaborativo con distintos departamentos, lo que enriqueció mi comprensión del negocio y la toma de decisiones estratégicas.


\section{Proyectos}
\subsection{Presentación para clientes B2B}

Más allá de las tareas diarias, que solían ser rutinarias y podían variar de un día a otro, mi proyecto más importante fue la creación de una presentación dinámica para las principales Key Accounts.

PedidosYa Envíos, de manera generalmente mensual, realiza reportes y presentaciones de desempeño para sus partners más importantes. El propósito de estas presentaciones es informar sobre los principales KPI, como el Fail Rate, Acceptance Rate y Vendor Late, tanto a nivel general del partner como desglosado por sucursal. Al recibir estas estadísticas, los partners pueden comprender mejor qué acciones tomar para optimizar sus operaciones. Dado que el servicio Courier funciona como una extensión del negocio del partner dentro de un modelo B2B, es también de interés para PedidosYa que sus partners mejoren su desempeño.

Para la implementación de esta solución, se utilizó gran parte de la Suite de Google, combinando \textbf{Google Slides}, \textbf{Google Sheets} y \textbf{BigQuery}. Se creó un pipeline de datos que extraía la información directamente desde la base de datos en BigQuery hacia Google Sheets. Allí, se aplicaban diversas tablas dinámicas para procesar y transformar la información en gráficos y tablas simplificadas, las cuales se integraban automáticamente en Google Slides. Este flujo de trabajo permitía que todos los datos se actualizaran automáticamente al inicio de cada mes, garantizando que la información presentada a los partners estuviera siempre actualizada y alineada con el último periodo evaluado.

La presentación debía ser concisa y efectiva, asegurando que la \textbf{Key Account Manager} (KAM) de Courier pudiera exponer los resultados en menos de 30 minutos. La estructura de la presentación se diseñó de la siguiente manera:

\begin{enumerate}
	\item \textbf{Resumen ejecutivo:} La primera diapositiva mostraba directamente los KPI principales y su variación respecto al mes anterior.
	\item \textbf{Evolución temporal:} Un \textbf{timeline} con la cantidad de órdenes y la variación del \textbf{Fail Rate} \textit{Day over Day} durante el mes.
	\item \textbf{Análisis de incidencias:} Un \textbf{Heatmap} de los principales tipos de incidencias que contribuyeron al \textbf{Fail Rate}.
	\item \textbf{Rendimiento por sucursal:} Dos diapositivas dedicadas a identificar las \textbf{mejores} y \textbf{peores} sucursales, mostrando su evolución según distintos KPI clave.
\end{enumerate}

Este proyecto permitió automatizar y optimizar la elaboración de reportes para los partners reduciendo el tiempo requerido para su preparación y mejorando la calidad de la información presentada. Además, reforzó la comunicación entre el equipo de Courier y los partners facilitando la toma de decisiones estratégicas basadas en datos.


\chapter[REFLECCIONES Y LECCIONES APRENDIDAS]{Reflecciones y lecciones aprendidas} \label{ch3}
\section{Aprendizaje}

Durante el período de práctica, adquirí y reforcé tanto habilidades técnicas como blandas. Gran parte de este aprendizaje fue posible gracias al equipo de Courier cuyo apoyo, paciencia y disposición para explicar tanto aspectos del negocio como herramientas utilizadas fueron fundamentales. Además, sus experiencias y conocimientos compartidos fueron valiosos y los considero un gran aporte para mi desarrollo profesional.

Uno de los aprendizajes más significativos fue comprender en profundidad el entorno laboral real enfrentándome a problemáticas concretas, responsabilidades en la toma de decisiones y la necesidad de una comunicación clara y efectiva. Aprendí a sintetizar información técnica de manera asertiva y accesible un aspecto clave en cualquier entorno empresarial. Asimismo, experimenté de primera mano la importancia de una buena dinámica de equipo y cómo la colaboración efectiva impacta en los resultados.

En cuanto a habilidades técnicas, adquirí conocimientos tanto en \textbf{análisis de datos} como en \textbf{negocios}. Estar presente en reuniones entre managers de distintas áreas y observar cómo los Account Managers negociaban con clientes me permitió comprender mejor el funcionamiento estratégico de la empresa. A nivel técnico, fortalecí mis conocimientos en \textbf{estadística y programación}, y entendí en profundidad la relevancia de \textbf{SQL} y las consultas avanzadas para la recopilación y análisis de datos.

\section{Oportunidades de mejora}

Desde el punto de vista de la experiencia profesional, la práctica fue altamente enriquecedora. El ambiente de trabajo fue exigente pero gratificante, y la empresa mantuvo un equilibrio entre el aprendizaje y la autonomía en las tareas asignadas. Además, la gestión por parte de la universidad para acceder a la práctica fue ágil y eficiente.

Si tuviera que mencionar un área de mejora, sería la falta de una guía previa que ayudara a los practicantes a anticipar ciertos aspectos relevantes para la elaboración del informe final. Por ejemplo, PedidosYa mantiene un alto nivel de confidencialidad respecto a sus datos y operaciones, lo que implicó restricciones estrictas en el acceso y manejo de información. Se me asignó un computador de la empresa con normas específicas para prevenir vulnerabilidades y filtraciones, lo que significó que no pude extraer información directamente para este informe. Aunque existía la posibilidad de tomar fotografías a la pantalla, decidí no hacerlo, ya que me parecía poco profesional y una práctica que podría interpretarse como una vulneración de las políticas de seguridad de la empresa.

Contar con una guía o una orientación previa sobre estas restricciones habría sido útil para planificar mejor la recopilación de información necesaria para el informe, dentro de los márgenes permitidos. No obstante, esta situación también me permitió desarrollar una mayor \textbf{conciencia sobre la gestión de datos sensibles} y la importancia de respetar las normativas empresariales en entornos corporativos.



\chapter[CONCLUSIONSES]{Conclusionses}
La experiencia en PedidosYa Envios marcó un antes y un después en mi desarrollo profesional y personal, permitiéndome aplicar y ampliar los conocimientos adquiridos en la carrera de Ingeniería Industrial. Durante mi práctica, tuve la oportunidad de enfrentar desafíos reales en un entorno laboral dinámico, lo que me impulsó a perfeccionar tanto mis habilidades técnicas como mis competencias interpersonales.

A lo largo de este periodo, comprendí la importancia de utilizar herramientas tecnológicas, como SQL, BigQuery y la Suite de Google, para transformar datos en insights que apoyen la toma de decisiones estratégicas. Este proceso no solo mejoró mi capacidad analítica, sino que también fortaleció mi compromiso con la precisión y la eficiencia en el manejo de la información.

Por otra parte, la experiencia me enseñó el valor de una comunicación clara y concisa en contextos colaborativos. La interacción con equipos multidisciplinarios y la participación en reuniones estratégicas me permitieron desarrollar la habilidad de presentar resultados y recomendaciones de forma accesible y orientada a objetivos comunes. Asimismo, me familiaricé con la realidad de operar bajo restricciones de seguridad y confidencialidad, lo que reforzó mi respeto por las buenas prácticas en la gestión de datos.

En definitiva, la práctica profesional en PedidosYa Envios no solo potenció mis competencias técnicas, sino que también me proporcionó una visión integral del funcionamiento de un negocio B2B en el sector logístico. Esta experiencia ha consolidado mi vocación y me ha preparado para enfrentar futuros desafíos con mayor seguridad y responsabilidad, constituyéndose como un pilar fundamental en mi formación profesional y personal.




%%%%%%%%%%%%% REFERENCES %%%%%%%%%%%%%

\cleardoublepage
\phantomsection \label{references}
\renewcommand{\bibname}{REFERENCIAS}

%%%% ACTIVAR SIGUIENTES 3 LINEAS SI POSTGRADO RECHAZA LA BIBLIOGRAFIA
%\setlength{\bibleftmargin}{0em}
%\setlength{\bibindent}{0em}
%\setlength{\bibitemsep}{1em}


\printbibliography

%%%%%%%%%%%% APPENDICES %%%%%%%%%%%%

\makeatletter
\renewcommand{\appendix}{%
  \setcounter{chapter}{0}%
  \renewcommand\@chapapp{\appendixname}%
  \renewcommand\thechapter{\@Alph\c@chapter}%
}
\makeatother

\appendix
\chapter[First Appendix]{First Appendix}
\begin{figure}[h!] % Or other placement options like t, b, p
\begin{flushleft}
  \includegraphics[width=\linewidth]{figures/autorizacion_practica.pdf}
\end{flushleft}
\caption{Autorizacion de practica en PedidosYa}
\label{fig: autorizacion_practica}
\end{figure}


\newpage
\section[Glossary]{Glossary}
\printglossary[title=Glossary]

\end{document}
