\section{Aprendizaje}

Durante el período de práctica, adquirí y reforcé tanto habilidades técnicas como blandas. Gran parte de este aprendizaje fue posible gracias al equipo de Courier cuyo apoyo, paciencia y disposición para explicar tanto aspectos del negocio como herramientas utilizadas fueron fundamentales. Además, sus experiencias y conocimientos compartidos fueron valiosos y los considero un gran aporte para mi desarrollo profesional.

Uno de los aprendizajes más significativos fue comprender en profundidad el entorno laboral real enfrentándome a problemáticas concretas, responsabilidades en la toma de decisiones y la necesidad de una comunicación clara y efectiva. Aprendí a sintetizar información técnica de manera asertiva y accesible un aspecto clave en cualquier entorno empresarial. Asimismo, experimenté de primera mano la importancia de una buena dinámica de equipo y cómo la colaboración efectiva impacta en los resultados.

En cuanto a habilidades técnicas, adquirí conocimientos tanto en \textbf{análisis de datos} como en \textbf{negocios}. Estar presente en reuniones entre managers de distintas áreas y observar cómo los Account Managers negociaban con clientes me permitió comprender mejor el funcionamiento estratégico de la empresa. A nivel técnico, fortalecí mis conocimientos en \textbf{estadística y programación}, y entendí en profundidad la relevancia de \textbf{SQL} y las consultas avanzadas para la recopilación y análisis de datos.

\section{Oportunidades de mejora}

Desde el punto de vista de la experiencia profesional, la práctica fue altamente enriquecedora. El ambiente de trabajo fue exigente pero gratificante, y la empresa mantuvo un equilibrio entre el aprendizaje y la autonomía en las tareas asignadas. Además, la gestión por parte de la universidad para acceder a la práctica fue ágil y eficiente.

Si tuviera que mencionar un área de mejora, sería la falta de una guía previa que ayudara a los practicantes a anticipar ciertos aspectos relevantes para la elaboración del informe final. Por ejemplo, PedidosYa mantiene un alto nivel de confidencialidad respecto a sus datos y operaciones, lo que implicó restricciones estrictas en el acceso y manejo de información. Se me asignó un computador de la empresa con normas específicas para prevenir vulnerabilidades y filtraciones, lo que significó que no pude extraer información directamente para este informe. Aunque existía la posibilidad de tomar fotografías a la pantalla, decidí no hacerlo, ya que me parecía poco profesional y una práctica que podría interpretarse como una vulneración de las políticas de seguridad de la empresa.

Contar con una guía o una orientación previa sobre estas restricciones habría sido útil para planificar mejor la recopilación de información necesaria para el informe, dentro de los márgenes permitidos. No obstante, esta situación también me permitió desarrollar una mayor \textbf{conciencia sobre la gestión de datos sensibles} y la importancia de respetar las normativas empresariales en entornos corporativos.

