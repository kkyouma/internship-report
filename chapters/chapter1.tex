\section{Descripción General de la Posición: Courier Business Intern \label{sec:sec1}}
El rol de \textit{Courier Business Intern} consiste en apoyar las labores del área de Courier Business, participando activamente en tareas de análisis de datos, operaciones y business intelligence. Este puesto complementa el trabajo del Sr. Analyst, colaborando en la recolección, análisis y presentación de información, así como en la implementación de soluciones operativas específicas para PedidosYa Envíos.

\section{Responsabilidades Clave \label{sec:sec2}}
Aunque el enfoque principal del rol es el análisis de datos y la toma de decisiones basadas en información, las responsabilidades se llevan a cabo en coordinación con el equipo de Courier Business. Es importante señalar que PedidosYa cuenta con equipos especializados (BI, Operaciones y Logística) para abordar estos desafíos de forma integral; sin embargo, para tareas y problemas específicos, se requiere una solución más focalizada dentro del equipo de Courier Business.

\subsection{Análisis de Datos y Reporte}
\begin{itemize}
  \item \textbf{Recolección y Procesamiento de Datos:} Extraer y organizar datos relevantes para evaluar el desempeño de las operaciones.
  \item \textbf{Búsqueda de Insights:} Identificar tendencias y oportunidades de mejora a partir de los datos analizados.
  \item \textbf{Generación y Presentación de Reportes:} Elaborar informes detallados que respalden la toma de decisiones estratégicas, en colaboración con el Sr. Analyst.
\end{itemize}

\subsection{Soporte en Operaciones}
\begin{itemize}
  \item \textbf{Tareas Operativas Específicas:} Realizar actividades no críticas como renombrar partners, activar cuentas o gestionar partners no clave (Key Accounts).
  \item \textbf{Implementación de Soluciones:} Colaborar en la aplicación de mejoras operativas que requieran un enfoque personalizado y granular.
\end{itemize}

\subsection{Presentación y Toma de Decisiones}
\begin{itemize}
  \item \textbf{Reuniones Semanales:} Preparar y exponer semanalmente reportes e insights ante el equipo y el Manager.
  \item \textbf{Sugerencias Estratégicas:} Proponer recomendaciones basadas en los análisis realizados para optimizar las operaciones y procesos.
\end{itemize}

\section{Herramientas y Tecnologías Utilizadas \label{sec:sec3}}
Para llevar a cabo las tareas descritas, PedidosYa utiliza diversas soluciones tecnológicas que facilitan la gestión de datos y la comunicación interna. Entre las principales herramientas se encuentran:

\begin{enumerate}
  \item \textbf{Jira:} Sistema de gestión de tickets utilizado para el escalado de incidencias de clientes y agentes.\footnote{Más información en: \url{https://www.atlassian.com/software/jira}}
  \item \textbf{Slack:} Plataforma de comunicación interna que permite una interacción fluida entre miembros de diferentes equipos.\footnote{Más información en: \url{https://slack.com/}}
  \item \textbf{Google Cloud Platform (GCP):} Infraestructura para la gestión de recursos tecnológicos y de datos.\footnote{Más información en: \url{https://cloud.google.com/gcp}}
  \item \textbf{BigQuery:} Servicio de almacenamiento y gestión de bases de datos SQL, parte de GCP, que facilita la extracción y análisis de información.\footnote{Más información en: \url{https://cloud.google.com/bigquery}}
  \item \textbf{Locker Studio:} Herramienta integrada en GCP para la creación de dashboards y consultas simples.
  \item \textbf{JupyterHub:} Plataforma IDE en línea para escribir código, principalmente en Python, utilizando librerías como Pandas y Matplotlib para la manipulación y visualización de datos.\footnote{Más información en: \url{https://jupyter.org/hub}}
  \item \textbf{Google Spreadsheets:} Aplicación similar a Excel para la creación de hojas de cálculo y tablas dinámicas, con funcionalidades de colaboración en tiempo real.\footnote{Más información en: \url{https://www.google.com/sheets/about/}}
\end{enumerate}


